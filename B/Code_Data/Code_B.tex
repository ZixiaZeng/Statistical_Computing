% Options for packages loaded elsewhere
\PassOptionsToPackage{unicode}{hyperref}
\PassOptionsToPackage{hyphens}{url}
%
\documentclass[
]{article}
\usepackage{amsmath,amssymb}
\usepackage{iftex}
\ifPDFTeX
  \usepackage[T1]{fontenc}
  \usepackage[utf8]{inputenc}
  \usepackage{textcomp} % provide euro and other symbols
\else % if luatex or xetex
  \usepackage{unicode-math} % this also loads fontspec
  \defaultfontfeatures{Scale=MatchLowercase}
  \defaultfontfeatures[\rmfamily]{Ligatures=TeX,Scale=1}
\fi
\usepackage{lmodern}
\ifPDFTeX\else
  % xetex/luatex font selection
\fi
% Use upquote if available, for straight quotes in verbatim environments
\IfFileExists{upquote.sty}{\usepackage{upquote}}{}
\IfFileExists{microtype.sty}{% use microtype if available
  \usepackage[]{microtype}
  \UseMicrotypeSet[protrusion]{basicmath} % disable protrusion for tt fonts
}{}
\makeatletter
\@ifundefined{KOMAClassName}{% if non-KOMA class
  \IfFileExists{parskip.sty}{%
    \usepackage{parskip}
  }{% else
    \setlength{\parindent}{0pt}
    \setlength{\parskip}{6pt plus 2pt minus 1pt}}
}{% if KOMA class
  \KOMAoptions{parskip=half}}
\makeatother
\usepackage{xcolor}
\usepackage[margin=1in]{geometry}
\usepackage{graphicx}
\makeatletter
\def\maxwidth{\ifdim\Gin@nat@width>\linewidth\linewidth\else\Gin@nat@width\fi}
\def\maxheight{\ifdim\Gin@nat@height>\textheight\textheight\else\Gin@nat@height\fi}
\makeatother
% Scale images if necessary, so that they will not overflow the page
% margins by default, and it is still possible to overwrite the defaults
% using explicit options in \includegraphics[width, height, ...]{}
\setkeys{Gin}{width=\maxwidth,height=\maxheight,keepaspectratio}
% Set default figure placement to htbp
\makeatletter
\def\fps@figure{htbp}
\makeatother
\setlength{\emergencystretch}{3em} % prevent overfull lines
\providecommand{\tightlist}{%
  \setlength{\itemsep}{0pt}\setlength{\parskip}{0pt}}
\setcounter{secnumdepth}{-\maxdimen} % remove section numbering
\ifLuaTeX
  \usepackage{selnolig}  % disable illegal ligatures
\fi
\usepackage{bookmark}
\IfFileExists{xurl.sty}{\usepackage{xurl}}{} % add URL line breaks if available
\urlstyle{same}
\hypersetup{
  pdftitle={Task B},
  pdfauthor={Zixia Zeng},
  hidelinks,
  pdfcreator={LaTeX via pandoc}}

\title{Task B}
\author{Zixia Zeng}
\date{2024-11-05}

\begin{document}
\maketitle

\subsection{B.1}\label{b.1}

\subsubsection{(1)}\label{section}

According to the background of B.1, the probability density function \$
p\_\lambda(x) \$ of a random variable \$ X \$ is: \[
p_\lambda(x) = \begin{cases} 
ae^{-\lambda(x - b)} & \text{if } x \geq b, \\
0 & \text{if } x < b,
\end{cases}
\] where: \$ b \textgreater{} 0 \$ is a known constant,
\$\lambda \textgreater{} 0 \$ is a parameter of the distribution, \$ a
\$ is a constant to be determined in terms of \$ \lambda \$ and \$ b \$.

According to the definition of probability density function, \$
p\_\lambda(x) \$ must integrate to 1 over its domain, so an equation can
be written: \[
\int_{-\infty}^{\infty} p_\lambda(x) \, dx = 1.
\] when \$ x \textless{} b \$, \$ p\_\lambda(x) = 0 \$, so we only need
to calculate the integral from \(x = b\) to \(x = \infty\). To set up
the integral, the equation can be written: \[
\int_{b}^{\infty} ae^{-\lambda(x - b)} \, dx = 1.
\] \(a\) is a constant number, so it can be factored out and just
calculate the remaining part: \[
a\int_{b}^{\infty} e^{-\lambda(x - b)} \, dx = 1.
\] Let \(\mu = x-b\), so domain changes to \(\{0,\infty\}\) and the
euqation should be transformed: \[
a\int_{0}^{\infty} e^{-\lambda\mu} \, d\mu = 1.
\] Solve the integral: \[
\begin{align*}
a\int_{0}^{\infty} e^{-\lambda\mu} d\mu 
&= -\frac{1}{\lambda} \cdot e^{-\lambda\mu}|_{0}^{\infty} \\
&= a \cdot [-\frac{1}{\infty} \cdot e^{-\infty \mu} -(-\frac{1}{0} \cdot e^{-0 \mu} )]\\
&= a \cdot [0-(-\frac{1}{\lambda})]\\
&= a \cdot \frac{1}{\lambda} = 1.
\end{align*}
\] so it is obvious that \[a = \lambda\] The answer of question(1) is
\(a = \lambda\)

\subsubsection{(2)}\label{section-1}

\end{document}
